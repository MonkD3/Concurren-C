\documentclass{beamer}
\usepackage[utf8]{inputenc}
\usepackage{amsmath}
\usepackage{amssymb}
\usepackage{multicol}

% C style code linting : https://tex.stackexchange.com/questions/348651/c-code-to-add-in-the-document
\usepackage{xcolor}
\usepackage{listings}

\definecolor{mGreen}{rgb}{0,0.6,0}
\definecolor{mGray}{rgb}{0.5,0.5,0.5}
\definecolor{mPurple}{rgb}{0.58,0,0.82}
\definecolor{backgroundColour}{rgb}{0.95,0.95,0.92}

\lstdefinestyle{CStyle}{
    backgroundcolor=\color{backgroundColour},
    commentstyle=\color{mGreen},
    keywordstyle=\color{magenta},
    numberstyle=\tiny\color{mGray},
    stringstyle=\color{mPurple},
    basicstyle=\footnotesize,
    breakatwhitespace=false,
    breaklines=true,
    captionpos=b,
    keepspaces=true,
    numbers=left,
    numbersep=5pt,
    showspaces=false,
    showstringspaces=false,
    showtabs=false,
    tabsize=2,
    language=C
}

% Change beamer theme to fit the needs
\usetheme{Boadilla}
\setbeamertemplate{footline}{}
\setbeamertemplate{itemize items}[default]
\setbeamertemplate{enumerate items}[default]


% Add COLORS
\newcommand{\textblue}[1]{\textbf{\textcolor{blue}{#1}}}


\title{Presentation of recent work}
\subtitle{Analysis of synchronization algorithm}
\author{Tihon Nathan}
\date{\today}

\addtobeamertemplate{navigation symbols}{}{%
    \usebeamerfont{footline}%
    \usebeamercolor[fg]{footline}%
    \hspace{1em}%
    \insertframenumber/\inserttotalframenumber%
}

\begin{document}

% Première page
\begin{frame}{\titlepage}
\end{frame}



\begin{frame}{The goal of the course}

  There was multiple topics to the course, although the following were the most important:

  \bigskip

  \begin{itemize}
    \item Study the roles and functions of an operating system
    \item Study the concepts, problems and solutions of processes and threads
    \item Usage of multiprocessor machines
  \end{itemize}

  \bigskip

  I like to think of this course as \textblue{a link between software and hardware}.

\end{frame}

\begin{frame}{Goal of the project}

  The goal of the project was to \textblue{write and analyse} the behaviour of 3 standard synchronization algorithms:

  \bigskip

  \begin{itemize}
    \item The Dining Philosophers problem
    \item The Producer-Consumer problem
    \item The Reader-Writer problem
  \end{itemize}

  \bigskip

  with regard to \textblue{the number of threads} and \textblue{the type of synchronization primitives}.

  \bigskip

  If we break it down to its first principles, the main goal of the project was to analyse the cost of the \texttt{xchg} instruction.

\end{frame}

\begin{frame}{Types of primitives}
  \begin{block}{Primitives}
    The comparison of synchronization primitives was done between:
    \begin{itemize}
            \item POSIX mutexes and semaphores
            \item My own ASM implementation of spinlock and semaphores built on top of them
    \end{itemize}
  \end{block}

  \bigskip

  I implemented two types of spinlock:

  \bigskip

  \begin{itemize}
    \item One based on the \textblue{test-and-set} algorithm
    \item The other based on the \textblue{test-and-test-and-set} algorithm
  \end{itemize}

\end{frame}

\begin{frame}{Results of the project}

  \begin{block}{Observation}
    The importants observations made during this project is that \textblue{\texttt{xchg} is a costly operation} and we must use it cautiously.
  \end{block}

  \bigskip

  \begin{block}{Performance}
    But active primitives perform better then POSIX primitives \textblue{when the waiting time is short}, this is because POSIX primitives are ''put on sleep'' and needs to be waken up, which also has a cost.
  \end{block}


\end{frame}


\begin{frame}{Naive implementation}

  \begin{block}{Simplest implementation}
    What most of my colleagues opted for is to duplicate the code and write each of the algorithms separately.
  \end{block}

  \bigskip

  While fast/easy to write, this method comes with a strong downside:

  \bigskip

  \begin{block}{Readability/extensibility}
    This method makes it very hard to navigate, read and extend the code.
  \end{block}

\end{frame}

\begin{frame}[containsverbatim]
\frametitle{Extensible implementation}

  The extensible implementation I opted for relies on the use of \textblue{unions nested in structs}

  \bigskip

  \begin{lstlisting}[style=CStyle]
typedef struct {
  algo_t type; // Type of mutex: 0=>POSIX, 1=>TAS, 2=>TATAS
  union {
    pthread_mutex_t* posix;
    spinlock_t* spinlock;
  }; // Use the type of mutex corresponding to type
} mutex_t;
  \end{lstlisting}

\end{frame}

\end{document}
